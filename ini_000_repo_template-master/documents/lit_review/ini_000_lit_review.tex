\documentclass{article}
\usepackage[utf8]{inputenc}
\usepackage[english]{babel}
\usepackage{csquotes}
\usepackage{lipsum}  
\usepackage{hyperref}
\hypersetup{
	colorlinks = true,
	citecolor = blue,
	urlcolor = blue
}
\urlstyle{same}

%%%%%%%%%%%%%%%%%%%%%%%%%%%%%%%%%%%%%%%
%% Bibliography
\newcommand{\Ncite}[1]{\citeauthor{#1} (\citeyear{#1})}
%Import the natbib package and sets a bibliography  and citation styles
\usepackage[backend=biber,style=authortitle]{biblatex}
\bibliography{ini_000_refs.bib}


\title{Literature Review Title}
\date{Date}
\author{author1, author2}
\begin{document}
	\maketitle
	
\section*{Summary}

\lipsum[2]



\section*{Annotated Bibliography}

\begin{enumerate}

	\item (\citeyear{schaink2012scoping}) \cite{schaink2012scoping} 
	\begin{itemize}
		\item Conducted literature review for definitions and descriptions of complexity and developed a Complexity Framework to understanding patient complexity.
		\item Used 1,669 articles, 127 of which were considered relevant as assessed by the inclusion/exclusion criteria. 
		\item Concluded the following complexity framework:
		\begin{enumerate}
			\item Medical/physical, e.g. functional impairment, chronic symptoms (e.g. pain), challenges in the application of clinical practice guidelines (CPGs), multimorbidity, and polypharmacy;
			\item Mental health, e.g. depression, psychological distress, cognitive impairment, and substance use;		
			\item Demographics, e.g. age, frailty, female, ethnicity, education;
			\item Social capital, e.g. caregiver strain, poor social support, relationship strain and lack of leisure time;		
			\item Health and social experiences, e.g. healthcare utilization, quality of life, self-management, healthcare system navigation.
		\end{enumerate}
	\end{itemize}

	\item (\citeyear{violan2014prevalence}) \cite{violan2014prevalence} 
	\begin{itemize} 
		\item Performed a systematic review of literature to determine if there was any established determinants for multimorbidity.
		\item Found age, lower socioeconomic status, and female gender as well as the presence of mental health problems to be well established determinants. 
		\item Found hypertension and osteoarthritis appear to be a very common combination followed by different cardiovascular condition combinations. 
		\item Concluded estimates of multimorbidity prevalence and the identification of specific patterns vary widely between studies. 
	\end{itemize}

 \end{enumerate}

%%%%%%%%%%%%%%%%%%%%%%%%%%%%%%%%%%%%%%%%%%%%%%%%%%%%%%%%%%%%%%%%%%%%%%%%%%%%%%%%%%%%%%%%%%%%%%%%%%%%%%%%%%%%%%%%%%%%%%%%%%%%%%%%%%%%%%%%%%%%%%%%%%%%%%%%%%%%%%%%%%%%%%%%%%%

\printbibliography

\end{document}          
