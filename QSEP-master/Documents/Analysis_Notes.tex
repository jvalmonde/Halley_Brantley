\documentclass[a4paper, 11pt]{article}
\usepackage{comment} % enables the use of multi-line comments (\ifx \fi) 
\usepackage{lipsum} %This package just generates Lorem Ipsum filler text. 
\usepackage{fullpage} % changes the margin
\usepackage{hyperref}
\usepackage{amsmath}

\begin{document}
	\noindent
	\normalsize
	\setlength{\parindent}{0em}
	\large\textbf{QSEP Notes} \hfill \textbf{Halley Brantley} \\
	
	\section*{Killer Questions}
	
	\begin{itemize}
 	\item Can we identify a dataset containing the pediatric illness populations of interest and their immediate family members? 
	\item Can we quantify the outcomes of interest and their relationship to the hypothesized mediating factors? 
	\item Do we see any significant secondary effects of pediatric illness on family members, in comparison to a sensibly-defined control group? 
	\end{itemize}
	
	\section*{Pediatric Illnesses}
	\begin{itemize}
		\item Cancer
		\item Autism
		\item Asthma
		\item Cerebral Palsy
		\item Traumatic Event - being re-calculated
	\end{itemize}
	
	\section*{Family Member Outcomes}
	\begin{itemize}
		\item	Incidence of mental health diagnoses (depression or anxiety) 
		\item	Incidence of infection or illness requiring hospitalization
		\item	Use of prescription drugs, specifically those for chronic pain 
		\item	Total cost of care 		
	\end{itemize}

\section*{Next steps}
\begin{itemize}
\item Look at the probability of a claim. Mixture distribution model on log scale given a claim, do you need health-care (yes/no), 

\item	2-part mixture model
\item	Specific mental health diagnoses
\item	Other predictors: median income, market segment, product description
\item	Other children as a covariate, indicator of other children. 
\item	Look at young parents with young children, difference between adult and child. 
\end{itemize}

CDF of a mixture model of a discrete probability at 0 and cdf $F_T$ for values of $Y > 0$. 
$$F(y) = \begin{cases} 
0 & y < 0 \\
1-p & y = 0 \\
1-p+ pF_T(y) & y > 0
\end{cases} $$

Define $\xi_i = \begin{cases} 1 & y_i > 0 \\ 0 & o.w. \end{cases}$  and $p = Pr(Y > 0)$

$$L(\theta|Y) = \prod_{i=1}^n p^\xi_i(1-p)^{1-\xi_i}f(y_i|\theta)^{\xi_i}$$

$$l(\theta|Y) = \sum_{i=1}^n \xi_i( \log p + \log f(y_i|\theta) ) + (1-\xi_i)log(1-p)$$

\url{https://cran.r-project.org/web/packages/gmm/vignettes/gmm_with_R.pdf}

Generalized method of moments: 
Given a vector of functions of $\theta_0$ with $E[g(\theta_0, x_i)] = 0$, there is often no solution to 
$$\bar{g}(\theta) = \frac{1}{n}\sum_{i=1}^n g(\theta, x_i)$$

We can make it as close to zero as possible by minimizing the quadratic function $g(\theta)^T W g(\theta)$, where $W$ is a positive definite and symmetric $q x q$ matrix of weights. The optimal weight matrix is the inverse of the asymptotic variance $\Omega(\theta_0)$. The optimal matrix can be estimated by an heteroskedascity and autocorrelation consistent matrix (HAC) whose general form is 
$$ \hat{\Omega} = \sum_{s=-(n-1)}^{n-1} k_h(s)\hat{\Gamma}_s(\theta^*)$$
where $k_h(s)$ is a kernel and $h$ is the bandwidth. 
$$\hat{\Gamma}_s(\theta^*) = \frac{1}{n}\sum_i g(\theta^*, x_i)g(\theta^*, x_{i+s})^T$$
Original GMM algorithm Hansen (1982)
\begin{enumerate}
	\item Compute $\theta^* = \arg \min \bar{g}(\theta)^T\bar{g}(\theta)$
	\item Compute the HAC matrix $\hat\Omega(\theta^*)$
	\item Compute the 2SGMM $\hat{\theta} \arg\min_{\theta}\bar{g}(\theta)^T[\hat\Omega(\theta^*)]^{-1}\bar{g}(\theta)$
\end{enumerate}

\end{document}	
