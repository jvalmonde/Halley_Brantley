%%%%%%%%%%%%%%%%%%%%%%%%%%%%%%%%%%%%%%%%%%%%%%%%%%%%%%%%%%%%
%%  This Beamer template was created by Cameron Bracken.
%%  Anyone can freely use or modify it for any purpose
%%  without attribution.
%%
%%  Last Modified: January 9, 2009
%%

\documentclass[xcolor=x11names,compress]{beamer}

%% General document %%%%%%%%%%%%%%%%%%%%%%%%%%%%%%%%%%
\usepackage{graphicx}
\usepackage{tikz}
\usepackage{graphicx}
\usepackage{amsmath}

\usetikzlibrary{decorations.fractals}
%%%%%%%%%%%%%%%%%%%%%%%%%%%%%%%%%%%%%%%%%%%%%%%%%%%%%%


%% Beamer Layout %%%%%%%%%%%%%%%%%%%%%%%%%%%%%%%%%%
\useoutertheme[subsection=false,shadow]{miniframes}
\useinnertheme{default}
\usefonttheme{serif}
\usepackage{palatino}

\setbeamerfont{title like}{shape=\scshape}
\setbeamerfont{frametitle}{shape=\scshape}

\setbeamercolor*{lower separation line head}{bg=DeepSkyBlue4} 
\setbeamercolor*{normal text}{fg=black,bg=white} 
\setbeamercolor*{alerted text}{fg=red} 
\setbeamercolor*{example text}{fg=black} 
\setbeamercolor*{structure}{fg=black} 

\setbeamercolor*{palette tertiary}{fg=black,bg=black!10} 
\setbeamercolor*{palette quaternary}{fg=black,bg=black!10} 

\renewcommand{\(}{\begin{columns}}
\renewcommand{\)}{\end{columns}}
\newcommand{\<}[1]{\begin{column}{#1}}
\renewcommand{\>}{\end{column}}
%%%%%%%%%%%%%%%%%%%%%%%%%%%%%%%%%%%%%%%%%%%%%%%%%%


\begin{document}


%%%%%%%%%%%%%%%%%%%%%%%%%%%%%%%%%%%%%%%%%%%%%%%%%%%%%%

\begin{frame}
\title{QSEP Research Update}
%\subtitle{SUBTITLE}
\author{
	Halley Brantley\\
	{\it Savvy Sherpa}\\
}
\date{
	\begin{tikzpicture}[decoration=Koch curve type 2] 
	\draw[DeepSkyBlue4] decorate{ decorate{ decorate{ (0,0) -- (3,0) }}}; 
	\end{tikzpicture}  
	\\
	\vspace{1cm}
	\today
}
\titlepage
\end{frame}

\section{\scshape Overview}

\begin{frame}{Research Questions}
Original
\begin{itemize}
	\item Can we identify a dataset containing the pediatric illness populations of interest and their immediate family members? 
	\item Can we quantify the outcomes of interest and their relationship to the hypothesized mediating factors? 
	\item Do we see any significant secondary effects of pediatric illness on family members, in comparison to a sensibly-defined control group? 
\end{itemize}
Updated 
\begin{itemize}
	\item Do we see a difference in adult family member spending before and after a child is diagnosed?
\end{itemize}
\end{frame}

\section{\scshape Data}

\begin{frame}{Dataset - from data team}
\begin{itemize}
	\item Count of sick children (under 24) in our dataset:  275,819
	\item Total member count (family members \& sick children combined): 968,471
	\item Total family count: 237,829
	\item Dataset covers year 2014-2016 and includes months since diagnosis, median household income, distance to PCP. 
	\item Between 65-70\% of members in each year were enrolled for 12 months
\end{itemize}
\end{frame}

\begin{frame}{Number of Children Under 16 with each condition}
\begin{itemize}
	\item Total sick children under 16: 159,792 
	\item Asthma:  99,065
	\item Autism (ASD): 10,966 
	\item Cancer: 29,572
	\item Cerebral Palsy: 1,802
	\item Type 1 Diabetes (T1D): 2,089
	\item Traumatic Event: 29,159
	\item Multiple Conditions: 12,142
\end{itemize}	
\end{frame}

\begin{frame}
Number of children under 16 with each condition, enrolled for more than 3 months, and for whom we have at least one month of spend data before diagnosis. 
\begin{itemize}
	\item Total sick children under 16: 136,791
	\item Unique Families: 122,588 
	\item Asthma:  87,872
	\item Autism (ASD): 8,732
	\item Cancer: 27,466
	\item Cerebral Palsy: 1,440
	\item Type 1 Diabetes (T1D): 1,434
	\item Traumatic Event: 22,313
	\item Multiple Conditions: 11,756
\end{itemize}	
\end{frame}

\begin{frame}{Data Cleaning }
\begin{itemize}
	\item To identify parents/guardians in the household used the criteria:
		\begin{itemize}
			\item Same household as sick child
			\item Older than 15
			\item At least 15 years older than the sick child 
			\item No more than 60 years older than the sick child
			\item Within 15 years of the age of the oldest member of the household
		\end{itemize} 
	\item Typically 1 or 2 members per household that meet these criteria, although several households have 3. 
\end{itemize}
\end{frame}


\section{\scshape Modeling}

\begin{frame}{Models}
\begin{itemize}
	\item Goal: model monthly spend per adult family member before and after diagnosis.
	\item Model 1: Treat spend as binary and model probability of non-zero spend
	\item Model 2: Look at only positive values of spend and model log total allowed amount. 
	\item Models were fit using generalized additive models to incorporate non-linear effects
\end{itemize}
\end{frame}

\begin{frame}{Predictors}
\begin{itemize}
\item Non-linear Predictors:  
\begin{itemize}
	\item Adult's age
	\item Sick child's age
	\item Median household income
\end{itemize}
\item Linear predictors:
\begin{itemize}
	\item Time
	\item RAF categories 
	\item Plan Type: Gated HMO, Choice Plus, Select Plus, Choice, Other
	\item Indicator of male
	\item Indicator of only one adult on the plan that meets the above criteria (no other parents on plan)
	\item Indicator of other children on the plan
	\item Interaction between single adult and other children
	\item Interaction between single adult and male
\end{itemize}  
\end{itemize}
\end{frame}

\begin{frame}
Predictors of interest:
\begin{itemize}
	\item Indicator of after child's diagnosis
	\item Interaction between after diagnosis and male
\end{itemize}
\end{frame}

\section{\scshape Results}



\begin{frame}
Factor increase in odds of positive monthly spend
\tiny
	%latex.default(oddsFactors_out[, 2:7], title = "", file = "../tables/oddsFactors.tex",     rowname = c("Intercept", "After child's diagnosis", "After child's diagnosis x Male",         "Male", "RAF (0.25, 1]", "RAF (1, 5]", "RAF (5, 10]",         "RAF (10, 100]", "s(Median Household Income)", "s(Age adult)",         "s(Sick child age)", "Single Adult", "Other Children",         "Time", "Choice Plus Plan", "Select Plus", "Choice",         "Gated HMO", "Male x RAF (0.25, 1]", "Male x RAF (1, 5]",         "Male x RAF (5, 10]", "Male x RAF (10, 100]", "Other Children x Single Adult",         "Male x Single Adult"), na.blank = FALSE)%
\begin{table}[!tbp]
\begin{center}
\begin{tabular}{lllllll}
\hline\hline
\multicolumn{1}{l}{}&\multicolumn{1}{c}{ASD}&\multicolumn{1}{c}{Asthma}&\multicolumn{1}{c}{Cancer}&\multicolumn{1}{c}{Cerebral}&\multicolumn{1}{c}{T1D}&\multicolumn{1}{c}{Trauma}\tabularnewline
\hline
Intercept&0.236&0.289&0.295&0.398&0.406&0.29\tabularnewline
After child's diagnosis&1.072&1.063&1.056&1.199&1.016&-\tabularnewline
After child's diagnosis x Male&1.055&1.03&-&-&1.219&1.066\tabularnewline
Male&0.446&0.397&0.41&0.355&0.369&0.4\tabularnewline
RAF (0.25, 1]&0.921&0.873&0.893&0.778&0.91&0.848\tabularnewline
RAF (1, 5]&3.716&3.002&3.182&3.031&4.384&3.296\tabularnewline
RAF (5, 10]&4.892&4.594&3.828&3.614&5.319&4.383\tabularnewline
RAF (10, 100]&5.737&4.766&4.219&7.053&12.113&5.195\tabularnewline
s(Median Household Income)&1.002&1.002&1.001&1.001&1.002&1.002\tabularnewline
s(Age adult)&1.029&1.031&1.031&1.024&1.015&1.027\tabularnewline
s(Sick child age)&1.01&0.992&0.992&0.991&-&0.994\tabularnewline
Single Adult&0.846&0.93&-&-&1.125&0.983\tabularnewline
Other Children&1.016&0.961&0.959&-&-&-\tabularnewline
Time&-&-&-&0.998&-&1.002\tabularnewline
Choice Plus Plan&1.163&1.134&1.183&1.094&1.131&1.171\tabularnewline
Select Plus&1.024&1.063&1.084&0.874&1.248&1.021\tabularnewline
Choice&1.07&1.065&1.149&0.955&1.185&1.095\tabularnewline
Gated HMO&0.955&0.95&1.013&0.961&0.95&0.912\tabularnewline
Male x RAF (0.25, 1]&1.194&1.329&1.31&1.735&1.437&1.362\tabularnewline
Male x RAF (1, 5]&1.31&1.524&1.48&2.123&1.712&1.56\tabularnewline
Male x RAF (5, 10]&1.057&1.269&1.485&1.944&1.777&1.509\tabularnewline
Male x RAF (10, 100]&0.847&1.392&1.444&0.695&0.569&1.489\tabularnewline
Other Children x Single Adult&-&1.039&-&1.276&0.838&1.032\tabularnewline
Male x Single Adult&0.944&0.916&0.951&0.829&0.798&0.873\tabularnewline
\hline
\end{tabular}\end{center}
\end{table}

	"-" indicates coefficient was not significant. 
\end{frame}

\begin{frame}
Effect of a diagnosis on men and women's odds of monthly spend
%latex.default(odds_AfterDiag, title = "", file = "../tables/Logistic_afterdiag.tex")%
\begin{table}[!tbp]
\begin{center}
\begin{tabular}{lll}
\hline\hline
\multicolumn{1}{l}{}&\multicolumn{1}{c}{Female}&\multicolumn{1}{c}{Male}\tabularnewline
\hline
ASD&1.072&1.212\tabularnewline
Asthma&1.063&1.164\tabularnewline
Cancer&1.056&1.115\tabularnewline
Cerebral&1.199&1.438\tabularnewline
T1D&1.016&1.258\tabularnewline
Trauma&1&1.066\tabularnewline
\hline
\end{tabular}\end{center}
\end{table}

\end{frame}

\begin{frame}
Factor increase in amount of monthly spend
\tiny
%latex.default(oddsFactors_out[, 2:7], title = "", file = "../tables/spendFactors.tex",     rowname = c("Intercept", "After child's diagnosis", "After child's diagnosis x Male",         "Male", "RAF (0.25, 1]", "RAF (1, 5]", "RAF (5, 10]",         "RAF (10, 100]", "s(Median Household Income)", "s(Age adult)",         "s(Sick child age)", "Single Adult", "Other Children",         "Time", "Choice Plus Plan", "Select Plus", "Choice",         "Gated HMO", "Male x RAF (0.25, 1]", "Male x RAF (1, 5]",         "Male x RAF (5, 10]", "Male x RAF (10, 100]", "Other Children x Single Adult",         "Male x Single Adult"), na.blank = FALSE)%
\begin{table}[!tbp]
\begin{center}
\begin{tabular}{lllllll}
\hline\hline
\multicolumn{1}{l}{}&\multicolumn{1}{c}{ASD}&\multicolumn{1}{c}{Asthma}&\multicolumn{1}{c}{Cancer}&\multicolumn{1}{c}{Cerebral}&\multicolumn{1}{c}{T1D}&\multicolumn{1}{c}{Trauma}\tabularnewline
\hline
Intercept&101.693&114.376&103.383&150.702&140.57&130.982\tabularnewline
After child's diagnosis&1.007&1.03&1&1.093&-&0.951\tabularnewline
After child's diagnosis x Male&-&-&-&-&-&1.112\tabularnewline
Male&0.813&0.817&0.795&0.827&0.864&0.775\tabularnewline
RAF (0.25, 1]&1.016&0.984&1.035&0.93&1.189&1.004\tabularnewline
RAF (1, 5]&1.913&1.904&1.931&2.168&2.654&1.945\tabularnewline
RAF (5, 10]&4.383&4.364&3.821&4.502&4.792&3.862\tabularnewline
RAF (10, 100]&8.69&6.484&5.935&6.866&11.901&8.089\tabularnewline
s(Median Household Income)&1.003&1.003&1.003&1.003&1.004&1.002\tabularnewline
s(Age adult)&-&1.001&1.005&0.991&0.997&0.999\tabularnewline
s(Sick child age)&-&0.993&0.989&-&0.991&0.995\tabularnewline
Single Adult&1.067&0.961&1.01&1.126&-&-\tabularnewline
Other Children&-&1.001&1.039&1.059&-&-\tabularnewline
Time&1.003&1.001&1.001&-&-&1.003\tabularnewline
Choice Plus Plan&1.147&1.047&1.019&0.9&0.796&0.992\tabularnewline
Select Plus&1.292&1.029&1.02&1.055&0.711&0.998\tabularnewline
Choice&1.165&1.118&1.084&0.989&0.838&1.049\tabularnewline
Gated HMO&1.011&0.951&0.965&0.863&0.629&0.902\tabularnewline
Male x RAF (0.25, 1]&0.91&0.99&0.998&1.332&0.923&0.982\tabularnewline
Male x RAF (1, 5]&1.067&1.001&1.073&1.109&1.197&1.071\tabularnewline
Male x RAF (5, 10]&0.88&0.86&0.953&1.633&1.463&0.945\tabularnewline
Male x RAF (10, 100]&0.874&1.076&1.198&1.052&1.032&0.868\tabularnewline
Other Children x Single Adult&-&1.03&0.935&-&-&-\tabularnewline
Male x Single Adult&-&1.026&-&-&-&-\tabularnewline
\hline
\end{tabular}\end{center}
\end{table}

"-" indicates coefficient was not significant. 
\end{frame}

\begin{frame}
Effect of a diagnosis on amount of men and women's monthly spend
%latex.default(odds_AfterDiag, title = "", file = "../tables/Spend_afterdiag.tex")%
\begin{table}[!tbp]
\begin{center}
\begin{tabular}{lll}
\hline\hline
\multicolumn{1}{l}{}&\multicolumn{1}{c}{Female}&\multicolumn{1}{c}{Male}\tabularnewline
\hline
ASD&1.007&1.007\tabularnewline
Asthma&1.03&1.03\tabularnewline
Cancer&1&1\tabularnewline
Cerebral&1.093&1.093\tabularnewline
T1D&1&1\tabularnewline
Trauma&0.951&1.058\tabularnewline
\hline
\end{tabular}\end{center}
\end{table}

\end{frame}

\begin{frame}
Combined Model - diagnosis terms
%latex.default(outFactors[1:13, ], rowname = outNames[1:13], title = "",     file = "../tables/combined_model_1.tex")%
\begin{table}[!tbp]
\begin{center}
\begin{tabular}{lrr}
\hline\hline
\multicolumn{1}{l}{}&\multicolumn{1}{c}{Logistic}&\multicolumn{1}{c}{Log-normal}\tabularnewline
\hline
(Intercept)&$0.289$&$114.740$\tabularnewline
After ASD&$1.119$&$  1.056$\tabularnewline
After Asthma&$1.079$&$  1.028$\tabularnewline
After Cancer&$1.131$&$  1.057$\tabularnewline
After Cerebral&$1.020$&$  0.985$\tabularnewline
After T1D&$0.965$&$  0.953$\tabularnewline
After Trauma&$0.964$&$  1.003$\tabularnewline
After ASD x male&$1.071$&$  0.983$\tabularnewline
After Asthma x male&$1.004$&$  1.011$\tabularnewline
After Cancer x male&$1.012$&$  1.045$\tabularnewline
After Cerebral  x male&$1.013$&$  1.129$\tabularnewline
After T1D x male&$1.110$&$  1.053$\tabularnewline
After Trauma x male&$1.042$&$  0.924$\tabularnewline
\hline
\end{tabular}\end{center}
\end{table}

\end{frame}

\begin{frame}
Combined Model - other terms
\tiny
%latex.default(outFactors[14:34, ], rowname = outNames[14:34],     title = "", file = "../tables/combined_model_2.tex")%
\begin{table}[!tbp]
\begin{center}
\begin{tabular}{lrr}
\hline\hline
\multicolumn{1}{l}{}&\multicolumn{1}{c}{Logistic}&\multicolumn{1}{c}{Log-normal}\tabularnewline
\hline
male&$0.406$&$  0.814$\tabularnewline
RAF (0.25, 1]&$0.873$&$  1.054$\tabularnewline
RAF (1, 5]&$3.112$&$  1.915$\tabularnewline
RAF (5, 10]&$4.347$&$  5.108$\tabularnewline
RAF (10, 100]&$4.794$&$  1.003$\tabularnewline
s(MedianHouseholdIncome)&$1.002$&$  1.001$\tabularnewline
s(Age Adult)&$1.030$&$  0.994$\tabularnewline
s(sick child age)&$0.993$&$  0.974$\tabularnewline
single adult&$0.953$&$  1.007$\tabularnewline
other child&$0.968$&$  1.001$\tabularnewline
time&$1.000$&$  1.042$\tabularnewline
Choice Plus Plan&$1.142$&$  1.047$\tabularnewline
Select Plus&$1.050$&$  1.109$\tabularnewline
Choice&$1.078$&$  0.946$\tabularnewline
Gated HMO&$0.963$&$  0.977$\tabularnewline
Male x RAF (0.25, 1]&$1.332$&$  1.032$\tabularnewline
Male x RAF (1, 5]&$1.517$&$  0.987$\tabularnewline
Male x RAF (5, 10]&$1.335$&$  1.014$\tabularnewline
Male x RAF (10, 100]&$1.365$&$  1.025$\tabularnewline
single adult x other child&$1.030$&$114.740$\tabularnewline
male x single adult&$0.916$&$  1.056$\tabularnewline
\hline
\end{tabular}\end{center}
\end{table}

\end{frame}
\end{document}